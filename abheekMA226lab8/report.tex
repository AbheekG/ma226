\documentclass{article}
\usepackage{amsfonts}
\usepackage{amsmath}
\usepackage{listings}
\usepackage{mybigpackage}
\usepackage{graphicx}

\begin{document}
	\title{\textbf{Assignment-8}}
	\author{Abheek Ghosh \\ 
		140123047 }
	
	\maketitle
	

\section{Question 1}

\noindent{Code for R}

\begin{lstlisting}
m <- 100
z <- 1.96	#for 95 \% confidence interval

while (m<=100000) {
	U <- runif(m)
	Y <- exp(U^(1/2))
	I <- mean(Y)
	SD <- sd(Y)
	cat("m = ", m,"\n")
	cat("Expected value, I = ", I, "\n")
	cat("Variance = ", SD^2, "\n")
	cat("95 \% confidence interval = (", I - z*SD/(m^(1/2)), ", ",I + z*SD/(m^(1/2)), ")\n\n")
	m <- m*10
}

rm(list = ls())
\end{lstlisting}
Using standard method\\\\
m =  100 \\
Expected value, I =  2.046535 \\
Variance =  0.182022 \\
95 \% confidence interval = ( 1.962914 ,  2.130157 )\\\\
m =  1000 \\
Expected value, I =  1.999513 \\
Variance =  0.2015493 \\
95 \% confidence interval = ( 1.971687 ,  2.027339 )\\\\
m =  10000 \\
Expected value, I =  2.006749 \\
Variance =  0.1921842 \\
95 \% confidence interval = ( 1.998157 ,  2.015341 )\\\\
m =  1e+05 \\
Expected value, I =  2.00195 \\
Variance =  0.1951535 \\
95 \% confidence interval = ( 1.999211 ,  2.004688 )\\
\pagebreak

\section{Question 2}

\noindent{Code for R}

\begin{lstlisting}
m <- 100
z <- 1.96	#for 95 \% confidence interval

while (m<=100000) {
	U <- runif(m)
	Y1 <- exp(U^(1/2))
	Y2 <- exp((1-U)^(1/2))
	Y <- (Y1 + Y2)/2
	I <- mean(Y)
	std_dev <- sd(Y)
	cat("m = ", m,"\n")
	cat("Expected value, I = ", I, "\n")
	cat("Variance = ", std_dev^2, "\n")
	cat("95 \% confidence interval = (", I - z*std_dev/(m^(1/2)), ", ",I + z*std_dev/(m^(1/2)), ")\n")
	cat("Variance reduction = ", 100*(1 - var(Y)/var(Y1)) ,"%\n\n")
	m <- m*10
}

rm(list = ls())
\end{lstlisting}
Using antithetic variate\\\\
m =  100 \\
Expected value, I =  2.004638 \\
Variance =  0.0008439335 \\
95 \% confidence interval = ( 1.998944 ,  2.010332 )\\
Variance reduction =  99.50857 \% \\\\
m =  1000 \\
Expected value, I =  2.001343 \\
Variance =  0.0009813989 \\
95 \% confidence interval = ( 1.999401 ,  2.003284 )\\
Variance reduction =  99.47258 \% \\\\
m =  10000 \\
Expected value, I =  1.999538 \\
Variance =  0.001085473 \\
95 \% confidence interval = ( 1.998892 ,  2.000184 )\\
Variance reduction =  99.44871 \% \\\\
m =  1e+05 \\
Expected value, I =  1.999922 \\
Variance =  0.001080687 \\
95 \% confidence interval = ( 1.999718 ,  2.000126 ) \\
Variance reduction =  99.44492 \% \\\\

\pagebreak

\section{Question 3}

\noindent{Code for R}

\begin{lstlisting}
m <- 100
z <- 1.96	#for 95 \% confidence interval

while (m<=100000) {
	U <- runif(m)
	Y1 <- exp(U^(1/2))

	X <- U
	X <- X^(1/2)
	c <- -cov(X,Y1)/var(X)
	mean_x <- mean(X)
	Y <- Y1 + c*(X - mean_x)
	I <- mean(Y)
	std_dev <- sd(Y)
	cat("m = ", m,"\n")
	cat("Expected value, I = ", I, "\n")
	cat("Variance = ", std_dev^2, "\n")
	cat("95 \% confidence interval = (", I - z*std_dev/(m^(1/2)), ", ",I + z*std_dev/(m^(1/2)), ")\n")
	cat("Variance reduction = ", 100*(1 - var(Y)/var(Y1)) ,"%\n\n")
	m <- m*10
}

rm(list = ls())
\end{lstlisting}

m =  100 \\
Expected value, I =  1.941239 \\
Variance =  0.002987869 \\
95 \% confidence interval = ( 1.930525 ,  1.951952 )\\
Variance reduction =  98.49517 \% \\\\
m =  1000 \\
Expected value, I =  1.992915 \\
Variance =  0.00266213 \\
95 \% confidence interval = ( 1.989718 ,  1.996113 )\\
Variance reduction =  98.58274 \% \\\\
m =  10000 \\
Expected value, I =  2.011091 \\
Variance =  0.002700158 \\
95 \% confidence interval = ( 2.010072 ,  2.012109 )\\
Variance reduction =  98.61463 \% \\\\
m =  1e+05 \\
Expected value, I =  2.000766 \\
Variance =  0.002692681 \\
95 \% confidence interval = ( 2.000444 ,  2.001088 )\\
Variance reduction =  98.61088 \% \\

\end{document}
