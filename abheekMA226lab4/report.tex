\documentclass{article}
\usepackage{amsfonts}
\usepackage{amsmath}
\usepackage{listings}
\usepackage{mybigpackage}
\usepackage{graphicx}

\begin{document}
	\title{\textbf{Assignment-4}}
	\author{Abheek Ghosh \\ 
		140123047 }
	
	\maketitle
	

\section{Question 1}

\noindent{Code for R}

\begin{lstlisting}
n <- 1000	 #No of values generated
c <- ((2 * exp(1)) / pi )^(1/2) 
x <- runif(n)
g <- vector(,n)
u <- runif(n)

for (i in 1:n) {
if (x[i] < 1/2) {
g[i] <- x[i]
x[i] <- log(2*x[i])
} else {
g[i] <- 1 - x[i]
x[i] <- -log(2*(1-x[i]))
}
}

f <- ((1 / (2 * pi)) ^ (1/2)) * (exp(-(x^2)/2))

rd <- x[(g*u*c) < f]

cat("Acceptance probability, Theoretical = ", 1/c, ", Stimulated = ", length(rd)/n, "\n")
cat("Mean, Theoretical = 0", ", Stimulated = ", mean(rd), "\n")
cat("Median, Theoretical = 0", "Stimulated = ", median(rd), "\n")
#cat("Mode, Theoretical = 0", "Stimulated = ", mode(rd), "\n")
cat("Variance, Theoretical = 1", ", Stimulated = ", var(rd), "\n")
cat("Standard Deviation, Theoretical = 1", ", Stimulated = ", sd(rd), "\n")

hist(rd, main="Standard Normal Distribution", xlab="Range of random numbers", ylab="Density")
dev.copy(png,"plot1.png");
dev.off ();
rm(list = ls())

\end{lstlisting}

We see that the random numbers generated have nearly same/approaching probability distribution, acceptance probability, mean, median, variance and standard deviation, to Standard Normal Distribution. So, we justify that generated random numbers are correct. For example, a sample of values generated have the values:

Acceptance probability, Theoretical =  0.7601735 , Stimulated =  0.763 \\
Mean, Theoretical = 0 , Stimulated =  -0.00010277 \\
Median, Theoretical = 0 Stimulated =  -0.00038851 \\
Variance, Theoretical = 1 , Stimulated =  0.9805156 \\
Standard Deviation, Theoretical = 1 , Stimulated =  0.9902099 \\


\includegraphics{"plot1"}
\pagebreak

\section{Question 2}

\noindent{Code for R}

\begin{lstlisting}
n <- 1000	 #No of values generated
c <- ((2 * exp(1)) / pi )^(1/2) 
x <- runif(n)
x <- -log(x)
u <- runif(n)

g <- exp(-x)
f <- ((2/pi) ^ (1/2)) * (exp(-(x^2)/2))

rd <- x[(g*c*u) < f]

cat("Acceptance probability, Theoretical = ", 1/c, ", Stimulated = ", length(rd)/n, "\n")
cat("Mean, Theoretical = ", (2/pi)^(1/2), ", Stimulated = ", mean(rd), "\n")
cat("Variance, Theoretical = ", 1 - (2/pi), ", Stimulated = ", var(rd), "\n")
cat("Standard Deviation, Theoretical = ", (1 - 2/pi)^(1/2), ", Stimulated = ", sd(rd), "\n")

hist(rd, main="Standard Half Normal Distribution", xlab="Range of random numbers", ylab="Density")
dev.copy(png,"plot2.png");
dev.off ();
rm(list = ls())
\end{lstlisting}

We see that the random numbers generated have nearly same/approaching probability distribution, acceptance probability, mean, median, variance and standard deviation, to Standard Half-Normal Distribution. So, we justify that generated random numbers are correct. For example, a sample of values generated have the values:

Acceptance probability, Theoretical =  0.7601735 , Stimulated =  0.762 \\
Mean, Theoretical =  0.7978846 , Stimulated =  0.7892713 \\
Variance, Theoretical =  0.3633802 , Stimulated =  0.3666513 \\
Standard Deviation, Theoretical =  0.6028103 , Stimulated =  0.6055174 \\
\includegraphics{"plot2"}
\pagebreak

\section{Question 3}

\noindent{Code for R, part a}

\begin{lstlisting}
#Discrete Inverse transformation method
n <- 10	 #No of values generated, taking 1000 as 10 is too small for any measurement
p <- c(0.05, 0.25, 0.45, 0.15, 0.10)
x <- c(1:5)

u <- runif(n)
rd <- vector(,n)

for (i in 1:n) {
for (j in 1:5) {
if (((sum(p[1:j]) - p[j]) < u[i]) & (u[i] < sum(p[1:j]))) {
rd[i] = j
}
}
}

print(rd)

cat("Mean, Theoretical = ", sum(p*x), ", Stimulated = ", mean(rd), "\n")
cat("Variance, Theoretical = ", (sum(p*x^2) - 9), ", Stimulated = ", var(rd), "\n")
hist(rd, main="Discrete Distribution", xlab="Range of random numbers", ylab="Density")
dev.copy(png,"plot3a.png");
dev.off ();
rm(list = ls())
\end{lstlisting}

Sample of 10 generated numbers are:
$$1, 3, 2, 4, 2, 3, 5, 2, 3, 3$$
Mean, Theoretical =  3 , Stimulated =  2.8 \\
Variance, Theoretical =  1 , Stimulated =  1.288889 \\

For a sample of 1000 numbers generated, the values are very close to theoretical.\\
Mean, Theoretical =  3 , Stimulated =  2.963 \\
Variance, Theoretical =  1 , Stimulated =  0.9946256 \\

\includegraphics{"plot3a"}

\noindent{Code for R, part b}

\begin{lstlisting}
#Discrete acceptance rejection method
n <- 20	 #No of values generated, taking 1000 as 10 is too small for any measurement
p <- c(0.05, 0.25, 0.45, 0.15, 0.10)
x <- c(1:5)
c <- max(p)/0.2

u <- runif(n)
v <- runif(n)
y <- as.integer(5*u) + 1

rd <- y[ v*0.45 < p[y] ]

print(rd)

cat("Mean, Theoretical = ", sum(p*x), ", Stimulated = ", mean(rd), "\n")
cat("Variance, Theoretical = ", (sum(p*x^2) - 9), ", Stimulated = ", var(rd), "\n")
hist(rd, main="Discrete Distribution", xlab="Range of random numbers", ylab="Density")
dev.copy(png,"plot3b.png");
dev.off ();
rm(list = ls())
\end{lstlisting}

Sample of 10 generated numbers are:
$$2, 5, 4, 3, 3, 2, 5, 3, 3, 2$$
Mean, Theoretical =  3 , Stimulated =  3.2 \\
Variance, Theoretical =  1 , Stimulated =  1.288889\\

For a sample of 1000 numbers generated, the values are very close to theoretical.\\
Mean, Theoretical =  3 , Stimulated =  3.055556 \\
Variance, Theoretical =  1 , Stimulated =  1.008675 \\

\includegraphics{"plot3b"}

\pagebreak

\end{document}
