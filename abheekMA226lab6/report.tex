\documentclass{article}
\usepackage{amsfonts}
\usepackage{amsmath}
\usepackage{listings}
\usepackage{mybigpackage}
\usepackage{graphicx}

\begin{document}
	\title{\textbf{Assignment-4}}
	\author{Abheek Ghosh \\ 
		140123047 }
	
	\maketitle
	

\section{Question 1}

\noindent{Code for R}

\begin{lstlisting}
#taking 3 p randomly.
q <- c(0.5, 0.8, 0.25)

#no of random numbers
n <- 50

u <- runif(n)

for (i in 1:3)
{
	r <- as.integer(log(u)/log(q[i])) + 1
	print(1-q[i])
	print(r)
	hist(r, main=paste("Geometric Distribution for about 50 values with p = ", 1-q[i]) , xlab="Range of random numbers", ylab="Density", breaks=50, probability=TRUE)
	if(i == 1)
		dev.copy(png, "plot1_1.png")
	if(i == 2)
		dev.copy(png, "plot1_2.png")
	if(i == 3)
		dev.copy(png, "plot1_3.png")
	dev.off ()
}
\end{lstlisting}
\pagebreak

Geometric Distribution

p = 0.5\\
Values\\
5 2 1 3 1 1 3 2 1 1 2 3 2 2 1 1 3 2 1 3 1 1 3 1 1 2 2 1 1 1 2 1 2 1 1 1 1 2
2 1 1 5 1 1 4 1 1 1 3 3\\\\
\includegraphics{"plot1_1"}
\pagebreak

p = 0.2\\
Values\\
14  5  1  8  1  1  8  7  3  3  5  7  4  5  3  3 10  4  2  7  3  2  8  1  2 
 4  5  2  1  2  4  1  5  3  1  2  2  5  4  2  1 14  1  2 13  3  3  2  9  8\\\\
\includegraphics{"plot1_2"}
\pagebreak

p = 0.75\\
Values\\
3 1 1 2 1 1 2 1 1 1 1 2 1 1 1 1 2 1 1 2 1 1 2 1 1 1 1 1 1 1 1 1 1 1 1 1 1 1 
1 1 1 3 1 1 2 1 1 1 2 2\\\\
\includegraphics{"plot1_3"}
\pagebreak

\section{Question 2}

\noindent{Code for R}

\begin{lstlisting}
n <- 50
lamda <- 2
u <- runif(n)
p0 <- exp(-lamda)
x <- vector(,n)
pms <- vector(,10)

for(j in 1:n)
{
	p <- p0
	f <- p
	i <- 0
	repeat {
		if(u[j] < f) {
			x[j] <- i
			if(i <= 10) {
				pms[i] <- pms[i] + 1
			}			
			break
		}
		p <- (lamda * p) / (i+1)
		f <- f + p
		i <- i + 1
	}
}

print(x[1:50])

pms <- pms/sum(pms)
cdf <- cumsum(pms)

hist(x, main="Poisson Distribution, mean = 2, 50 values", xlab="Range of random numbers", ylab="Density")
dev.copy(png,"plot2a.png");
dev.off ();

plot(1:10, pms, col='black', cex=1, main="Poisson Distribution, mean = 2, 50 values", xlab="Range of random numbers", ylab="Probability Mass Function")
dev.copy(png,"plot2b.png");
dev.off ();

plot(1:10, cdf, col='black', cex=1, main="Poisson Distribution, mean = 2, 50 values", xlab="Range of random numbers", ylab="Cumulative distribution Function")
dev.copy(png,"plot2c.png");
dev.off ();
\end{lstlisting}
Poisson Distribution with mean = 2.

Generated random numbers\\
1 1 2 3 2 0 2 0 3 2 1 2 1 5 3 1 0 3 3 6 4 4 3 2 1 2 0 1 3 0 3 1 1 1 2 3 0 3 0 5 1 5 1 3 3 1 2 3 2 3

\includegraphics{"plot2a"}
\pagebreak

\includegraphics{"plot2b"}
\pagebreak

\includegraphics{"plot2c"}
\pagebreak

\section{Question 3}

\noindent{Code for R}

\begin{lstlisting}
weibull <- function(u, b, t) {
	return ( exp( (log(-log(1-u))/b) - log(t) ))
}

b1 <- 2
t1 <- 1
b2 <- 1.5
t2 <- 1
p <- 0.4

n <- 50

u1 <- runif(n)
u2 <- runif(n)
x <- vector(,n)

for (i in 1:n)
{
	if(u1[i] < p) {
		x[i] <- weibull(u2[i], b1, t1)
	} else {
		x[i] <- weibull(u2[i], b2, t2)
	}
}

print(x)

hist(x, main="Mixed Weibull Distribution, parameters (2, 1, 1.5, 1, 0.4), 50 values", xlab="Range of random numbers", ylab="Density")
dev.copy(png,"plot3.png");
dev.off ();
\end{lstlisting}

Weibull Transformation.\\
$$\beta_1=2,\ \theta_1=1,\ \beta_2=1.5,\ \theta_2=1,\ p=0.4$$
Values generated\\
0.51340342 1.65981104 0.35708428 0.33864543 0.90015026 0.48173004\\
0.90146400 1.35998194 1.05895335 0.72202120 1.74569550 0.41448123\\
1.28357966 1.59452058 0.59331917 0.66916975 0.78349218 0.18203587\\
0.35765881 1.88747443 1.21824326 0.75508734 0.95769445 1.12843392\\
2.35256249 1.18500213 1.44337543 1.11540900 0.20877404 0.51849504\\
0.96475145 2.45482151 0.02652456 0.93861928 0.60031390 0.67181850\\
0.19249440 1.16159278 0.30561273 0.93054568 1.20750012 0.58390542\\
0.84283298 0.33951827 0.96317679 0.93258008 0.99216777 0.44624422\\
1.27564358 1.38776060

Histogram

\includegraphics{"plot3"}
\pagebreak
\end{document}
