\documentclass{article}
\usepackage{amsfonts}
\usepackage{amsmath}
\usepackage{listings}
\usepackage{mybigpackage}
\usepackage{graphicx}
\graphicspath{ {R/} {cpp/} }

\begin{document}
	\title{\textbf{Assignment-3}}
	\author{Abheek Ghosh \\ 
		140123047 }
	
	\maketitle
	

\section{Question 1}

\noindent{Code for C++}

\begin{lstlisting}
#include <iostream>
#include <iomanip>
#include <cmath>
#include <cstdlib>

#define RANGE 100

using namespace std;

class LCG {
private:
long a, b, m, x, q, r;
public:
LCG() {
a = 1103515245;
b = 12345;
m = 2147483648;
q=m/a;
r=m%a;
x = std::rand()/32768;
}
long base_generator() {
long k=x/q;
x = (a * (x - (k * q))) - (k * r);
x = (x + b) % m;
while(x<0) x+=m;
return x;
}
double generate() {
return (double)base_generator()/m;
}
void set_all(long sa, long sb, long sm, long sx) {
a = sa;
b = sb;
m = sm;
q=m/a;
r=m%a;
x = sx;
}
void set_x(long sx) {
x = sx;
}
};

double inv_exp_cdf(double x, double lamda) {
return (-log(x)/lamda);
}

int main() {
int n = 5000, Density[RANGE];
double given_mean = 5;
double lamda = 1/given_mean;
double mean=5, emax = 0, emin = 1.79769e+30, x;
LCG lcg;

for(int i = 0; i < RANGE; ++i) {
Density[i] = 0;
}

for(int i = 0; i < n; ++i) {
x = inv_exp_cdf(lcg.generate(), lamda);
//	cout<<x<<" ";
mean = ((mean * i) + x)/(i+1);
if(emin > x)
emin = x;
if(emax < x)
emax = x;
if(x*5 <= RANGE) {
Density[(int)(x*5)]++;
}
}
cout<<"#Mean = "<<mean<<endl;
cout<<"#Minimum = "<<emin<<endl;
cout<<"#Maximum = "<<emax<<endl;
for(int i = 0; i < RANGE; ++i) {
if(i%5==0) 
cout<<i/5<<" "<<Density[i]<<endl;
else
cout<<". "<<Density[i]<<endl;
}
return 0;
}
\end{lstlisting}

\noindent{Code for R}

\begin{lstlisting}
rd <- runif(5000)
rd <- -log(rd)*5
cat("Mean = ", mean(rd), "\n")
cat("Minimum = ", min(rd), "\n")
cat("Maximum = ", max(rd), "\n")
hist(rd, main="Exponential distribution with mean = 5", xlab="Range of random numbers", ylab="Density")
dev.copy(png,"plot1.png");
dev.off ();
rm(list = ls())
\end{lstlisting}

Simulate 5000 sample of exponential with mean 5. Draw the histogram and calculate the mean, maximum and minimum.\\\\
Mean =  4.993371 \\
Minimum =  0.0003285686 \\
Maximum =  39.08356 \\
\includegraphics[scale=0.3]{"q1"}

\includegraphics{"plot1"}
\pagebreak

\section{Question 2}
\noindent{Code for C++}

\begin{lstlisting}
#include <iostream>
#include <iomanip>
#include <cmath>
#include <cstdlib>

#define RANGE 90

using namespace std;

class LCG {
private:
long a, b, m, x, q, r;
public:
LCG() {
a = 1103515245;
b = 12345;
m = 2147483648;
q=m/a;
r=m%a;
x = std::rand()/32768;
}
long base_generator() {
long k=x/q;
x = (a * (x - (k * q))) - (k * r);
x = (x + b) % m;
while(x<0) x+=m;
return x;
}
double generate() {
return (double)base_generator()/m;
}
void set_all(long sa, long sb, long sm, long sx) {
a = sa;
b = sb;
m = sm;
q=m/a;
r=m%a;
x = sx;
}
void set_x(long sx) {
x = sx;
}
};

double inv_exp_cdf(double x, double lamda) {
return (-log(x)/lamda);
}

int main() {
int n = 5000, Density[RANGE];
double lamda = 5;
double mean=5, emax = 0, emin = 1.79769e+30, x;
LCG lcg;

for(int i = 0; i < RANGE; ++i) {
Density[i] = 0;
}

for(int i = 0; i < n; ++i) {
x = 0;
for(int j=0; j<5; ++j) {
x += inv_exp_cdf(lcg.generate(), lamda);
} 
//	cout<<x<<" ";
mean = ((mean * i) + x)/(i+1);
if(emin > x)
emin = x;
if(emax < x)
emax = x;
if(x*30 <= RANGE) {
Density[(int)(x*30)]++;
}
}
cout<<"Mean = "<<mean<<endl;
cout<<"Minimum = "<<emin<<endl;
cout<<"Maximum = "<<emax<<endl;
for(int i = 0; i < RANGE; ++i) {
if(i%30==0) 
cout<<i/30<<" "<<Density[i]<<endl;
else
cout<<". "<<Density[i]<<endl;
}
return 0;
}
\end{lstlisting}

\noindent{Code for R}

\begin{lstlisting}
rd1 <- runif(5000)
rd2 <- runif(5000)
rd3 <- runif(5000)
rd4 <- runif(5000)
rd5 <- runif(5000)

rd <- -(log(rd1) + log(rd2) + log(rd3) + log(rd4) + log(rd5))/5
cat("Mean = ", mean(rd), "\n")
cat("Minimum = ", min(rd), "\n")
cat("Maximum = ", max(rd), "\n")
hist(rd, main="Gamma distribution with N = 5 and Lamda = 5", xlab="Range of random numbers", ylab="Density")
dev.copy(png,"plot2.png");
dev.off ();
rm(list = ls())
\end{lstlisting}
Generate 5000 sample from Gamma with parameter $n = 5$ and
$ \lambda  = 5$. Draw the histogram and the calculate the mean, maximum and minimum.\\\\
Mean =  0.9943813 \\
Minimum =  0.07742613 \\
Maximum =  3.659501 \\
\includegraphics[scale=0.3]{"q2"}
\includegraphics{"plot2"}
\pagebreak

\section{Question 3}


\noindent{Code for R}

\begin{lstlisting}
#c <- 2.109376
rd1 <- runif(50000) #Taking g(x) as uniform dist rd1 is c*g(x)
rd2 <- runif(50000, 0, 2.109375)
fx <- 20 * rd1 * (1 - rd1)^3

rd <- rd1[rd2 < fx]

cat("Mean = ", mean(rd), "\n")
cat("Minimum = ", min(rd), "\n")
cat("Maximum = ", max(rd), "\n")
hist(rd, main="Distribution with f(x) = 20x(1-x)^3", xlab="Range of random numbers", ylab="Density")
dev.copy(png,"plot3.png");
dev.off ();
rm(list = ls())
\end{lstlisting}

Given in the question, density function:

$$f(x) = 20x(1-x)^3 \ \ 0<x<1$$
Mean =  0.3389358 \\
Minimum =  0.00600503 \\
Maximum =  0.9436907 \\

\includegraphics{"plot3"}
\pagebreak

\end{document}
