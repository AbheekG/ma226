\documentclass{article}
\usepackage{amsfonts}
\usepackage{amsmath}
\usepackage{listings}
\usepackage{mybigpackage}
\usepackage{graphicx}
\graphicspath{ {R/} {cpp/} }

\begin{document}
	\title{\textbf{Assignment-5}}
	\author{Abheek Ghosh \\ 
		140123047 }
	
	\maketitle
	

\section{Box-Muller Code}

\noindent{Code for R}

\begin{lstlisting}
start.time <- Sys.time()
#Box Muller method
n <- 10000 #no of values
u1 <- runif(n)
u2 <- runif(n)

temp <- ((-2) * log(u1))^(1/2)

z1 <- temp * cos(2 * pi * u2)
z2 <- temp * sin(2 * pi * u2)

end.time <- Sys.time()
time.taken <- end.time - start.time


cat("Box-Muller Method\n")
cat("Time taken for 10000 numbers generated = ", time.taken,"\n")
cat("For 100 values\n")
cat("Sample's Mean, X = ", mean(z1[1:100]), ", Y = ", mean(z2[1:100]), "\n")
cat("Sample's Variance, X = ", var(z1[1:100]), ", Y = ", var(z2[1:100]), "\n")
cat("Sample's Covariance = ", cov(z1[1:100], z2[1:100]), "\n\n")

cat("For 500 values\n")
cat("Sample's Mean, X = ", mean(z1[1:500]), ", Y = ", mean(z2[1:500]), "\n")
cat("Sample's Variance, X = ", var(z1[1:500]), ", Y = ", var(z2[1:500]), "\n")
cat("Sample's Covariance = ", cov(z1[1:500], z2[1:500]), "\n\n")

cat("For 10000 values\n")
cat("Sample's Mean, X = ", mean(z1), ", Y = ", mean(z2), "\n")
cat("Sample's Variance, X = ", var(z1), ", Y = ", var(z2), "\n")
cat("Sample's Covariance = ", cov(z1, z2), "\n\n")

plot(z1[1:100], z2[1:100], col='black',  main="2-D Standard Normal Distribution, 100 values,", xlab="X", ylab="Y")
dev.copy(png,"plot1_1.png");
dev.off ();

hist(z1[1:100], main="Standard Normal Distribution, X, 100 values", xlab="Range of random numbers", ylab="Density", breaks=15)
dev.copy(png,"plot1_1_X.png");
dev.off ();

hist(z2[1:100], main="Standard Normal Distribution, Y, 100 values", xlab="Range of random numbers", ylab="Density", breaks=15)
dev.copy(png,"plot1_1_Y.png");
dev.off ();



plot(z1[1:500], z2[1:500], col='black', cex=0.5, main="2-D Standard Normal Distribution, 500 values", xlab="X", ylab="Y")
dev.copy(png,"plot1_2.png");
dev.off ();

hist(z1[1:500], main="Standard Normal Distribution, X, 500 values", xlab="Range of random numbers", ylab="Density", breaks=20)
dev.copy(png,"plot1_2_X.png");
dev.off ();

hist(z2[1:500], main="Standard Normal Distribution, Y, 500 values", xlab="Range of random numbers", ylab="Density", breaks=20)
dev.copy(png,"plot1_2_Y.png");
dev.off ();



plot(z1, z2, pch=16, col='black', cex=0.5, main="2-D Standard Normal Distribution, 10000 values", xlab="X", ylab="Y")
dev.copy(png,"plot1_3.png");
dev.off ();

hist(z1, main="Standard Normal Distribution, X, 10000 values", xlab="Range of random numbers", ylab="Density", breaks=25)
dev.copy(png,"plot1_3_X.png");
dev.off ();

hist(z2, main="Standard Normal Distribution, Y, 10000 values", xlab="Range of random numbers", ylab="Density", breaks=25)
dev.copy(png,"plot1_3_Y.png");
dev.off ();


norm <- function(x, mu, sig) {
	return ( (1/ ((2*pi)^(1/2) * sig )) * exp ( -((x-mu)/sig)^2 / 2) )
}


##For N(μ= 0, σ=5)
x1 <- z1[1:500] * 5^(1/2)
y1 <- z2[1:500] * 5^(1/2)

cat("N(μ= 0, σ=5) for 500 values\n")
cat("Sample's Mean, X = ", mean(x1[1:500]), ", Y = ", mean(y1[1:500]), "\n")
cat("Sample's Variance, X = ", var(x1[1:500]), ", Y = ", var(y1[1:500]), "\n")
cat("Sample's Covariance = ", cov(x1[1:500], y1[1:500]), "\n\n")

plot(x1[1:500], y1[1:500], col='black', cex=0.5, main="2-D N(μ= 0, σ=5), 500 values", xlab="X", ylab="Y")
dev.copy(png,"plot3_1.png");
dev.off ();

hist(x1[1:500], main="N(μ= 0, σ=5), X, 500 values", xlab="Range of random numbers", ylab="Density", breaks=20)
points(x1 , 500 * norm(x1, 0, 5) , col='black', cex=0.5)
dev.copy(png,"plot3_1_X.png");
dev.off ();

hist(y1[1:500], main="N(μ= 0, σ=5), Y, 500 values", xlab="Range of random numbers", ylab="Density", breaks=20)
points(y1 , 500 * norm(y1, 0, 5) , col='black', cex=0.5)
dev.copy(png,"plot3_1_Y.png");
dev.off ();


##For N(μ= 5, σ=5)



x2 <- x1 + 5
y2 <- y1 + 5

cat("N(μ= 5, σ=5) for 500 values\n")
cat("Sample's Mean, X = ", mean(x2[1:500]), ", Y = ", mean(y2[1:500]), "\n")
cat("Sample's Variance, X = ", var(x2[1:500]), ", Y = ", var(y2[1:500]), "\n")
cat("Sample's Covariance = ", cov(x2[1:500], y2[1:500]), "\n\n")

plot(x2[1:500], y2[1:500], col='black', cex=0.5, main="2-D N(μ= 5, σ=5), 500 values", xlab="X", ylab="Y")
dev.copy(png,"plot3_2.png");
dev.off ();

hist(x2[1:500], main="N(μ= 5, σ=5), X, 500 values", xlab="Range of random numbers", ylab="Density", breaks=20)
points(x2 , 500 * norm(x2, 5, 5) , col='black', cex=0.5)
dev.copy(png,"plot3_2_X.png");
dev.off ();

hist(y2[1:500], main="N(μ= 5, σ=5), Y, 500 values", xlab="Range of random numbers", ylab="Density", breaks=20)
points(y2 , 500 * norm(y2, 5, 5) , col='black', cex=0.5)
dev.copy(png,"plot3_2_Y.png");
dev.off ();
rm(list = ls())
\end{lstlisting}

\section{Marsaglia Bray Code}

\noindent{Code for R}

\begin{lstlisting}
#Marsaglia Bray method
start.time <- Sys.time()
n <- 11000 	#no of values
p <- 0
q <- 0
u1 <- vector(,0)
u2 <- vector(,0)

while (q < 10000) {
	p <- p + n
	u <- runif(n)
	v <- runif(n)
	u <- 2*u - 1
	v <- 2*v - 1
	temp <- u^2 + v^2
	u1 <- c( u1, u[temp < 1] )
	u2 <- c( u2, v[temp < 1] )
	q <- length(u1)
	n <- 10000 - q + 1
}

temp1 <- u1[1:1000]^2 + u2[1:1000]^2
temp2 <- ((-2) * log(temp1))^(1/2)
temp1 <- temp1^(1/2)

z1 <- temp2 * (u1[1:1000] / temp1)
z2 <- temp2 * (u2[1:1000] / temp1)

end.time <- Sys.time()
time.taken <- end.time - start.time
time.taken


cat("Marsaglia-Bray Method\n")
cat("Time taken for 10000 numbers generated = ", time.taken,"\n")
cat("Rejection ratio, Theoretical = ", 1 - pi/4, ", Generated = ", 1 - q/p, "\n\n")
cat("For 100 values\n")
cat("Sample's Mean, X = ", mean(z1[1:100]), ", Y = ", mean(z2[1:100]), "\n")
cat("Sample's Variance, X = ", var(z1[1:100]), ", Y = ", var(z2[1:100]), "\n")
cat("Sample's Covariance = ", cov(z1[1:100], z2[1:100]), "\n\n")

cat("For 500 values\n")
cat("Sample's Mean, X = ", mean(z1[1:500]), ", Y = ", mean(z2[1:500]), "\n")
cat("Sample's Variance, X = ", var(z1[1:500]), ", Y = ", var(z2[1:500]), "\n")
cat("Sample's Covariance = ", cov(z1[1:500], z2[1:500]), "\n\n")

cat("For 10000 values\n")
cat("Sample's Mean, X = ", mean(z1), ", Y = ", mean(z2), "\n")
cat("Sample's Variance, X = ", var(z1), ", Y = ", var(z2), "\n")
cat("Sample's Covariance = ", cov(z1, z2), "\n\n")

plot(z1[1:100], z2[1:100], col='black',  main="2-D Standard Normal Distribution, 100 values,", xlab="X", ylab="Y")
dev.copy(png,"plot2_1.png");
dev.off ();

hist(z1[1:100], main="Standard Normal Distribution, X, 100 values", xlab="Range of random numbers", ylab="Density", breaks=15)
dev.copy(png,"plot2_1_X.png");
dev.off ();

hist(z2[1:100], main="Standard Normal Distribution, Y, 100 values", xlab="Range of random numbers", ylab="Density", breaks=15)
dev.copy(png,"plot2_1_Y.png");
dev.off ();



plot(z1[1:500], z2[1:500], col='black', cex=0.5, main="2-D Standard Normal Distribution, 500 values", xlab="X", ylab="Y")
dev.copy(png,"plot2_2.png");
dev.off ();

hist(z1[1:500], main="Standard Normal Distribution, X, 500 values", xlab="Range of random numbers", ylab="Density", breaks=20)
dev.copy(png,"plot2_2_X.png");
dev.off ();

hist(z2[1:500], main="Standard Normal Distribution, Y, 500 values", xlab="Range of random numbers", ylab="Density", breaks=20)
dev.copy(png,"plot2_2_Y.png");
dev.off ();



plot(z1, z2, pch=16, col='black', cex=0.5, main="2-D Standard Normal Distribution, 10000 values", xlab="X", ylab="Y")
dev.copy(png,"plot2_3.png");
dev.off ();

hist(z1, main="Standard Normal Distribution, X, 10000 values", xlab="Range of random numbers", ylab="Density", breaks=25)
dev.copy(png,"plot2_3_X.png");
dev.off ();

hist(z2, main="Standard Normal Distribution, Y, 10000 values", xlab="Range of random numbers", ylab="Density", breaks=25)
dev.copy(png,"plot2_3_Y.png");
dev.off ();


norm <- function(x, mu, sig) {
	return ( (1/ ((2*pi)^(1/2) * sig )) * exp ( -((x-mu)/sig)^2 / 2) )
}


##For N(μ= 0, σ=5)
x1 <- z1[1:500] * 5^(1/2)
y1 <- z2[1:500] * 5^(1/2)

cat("N(μ= 0, σ=5) for 500 values\n")
cat("Sample's Mean, X = ", mean(x1[1:500]), ", Y = ", mean(y1[1:500]), "\n")
cat("Sample's Variance, X = ", var(x1[1:500]), ", Y = ", var(y1[1:500]), "\n")
cat("Sample's Covariance = ", cov(x1[1:500], y1[1:500]), "\n\n")

plot(x1[1:500], y1[1:500], col='black', cex=0.5, main="2-D N(μ= 0, σ=5), 500 values", xlab="X", ylab="Y")
dev.copy(png,"plot4_1.png");
dev.off ();

hist(x1[1:500], main="N(μ= 0, σ=5), X, 500 values", xlab="Range of random numbers", ylab="Density", breaks=20)
points(x1 , 500 * norm(x1, 0, 5) , col='black', cex=0.5)
dev.copy(png,"plot4_1_X.png");
dev.off ();

hist(y1[1:500], main="N(μ= 0, σ=5), Y, 500 values", xlab="Range of random numbers", ylab="Density", breaks=20)
points(y1 , 500 * norm(y1, 0, 5) , col='black', cex=0.5)
dev.copy(png,"plot4_1_Y.png");
dev.off ();


##For N(μ= 5, σ=5)



x2 <- x1 + 5
y2 <- y1 + 5

cat("N(μ= 5, σ=5) for 500 values\n")
cat("Sample's Mean, X = ", mean(x2[1:500]), ", Y = ", mean(y2[1:500]), "\n")
cat("Sample's Variance, X = ", var(x2[1:500]), ", Y = ", var(y2[1:500]), "\n")
cat("Sample's Covariance = ", cov(x2[1:500], y2[1:500]), "\n\n")

plot(x2[1:500], y2[1:500], col='black', cex=0.5, main="2-D N(μ= 5, σ=5), 500 values", xlab="X", ylab="Y")
dev.copy(png,"plot4_2.png");
dev.off ();

hist(x2[1:500], main="N(μ= 5, σ=5), X, 500 values", xlab="Range of random numbers", ylab="Density", breaks=20)
points(x2 , 500 * norm(x2, 5, 5) , col='black', cex=0.5)
dev.copy(png,"plot4_2_X.png");
dev.off ();

hist(y2[1:500], main="N(μ= 5, σ=5), Y, 500 values", xlab="Range of random numbers", ylab="Density", breaks=20)
points(y2 , 500 * norm(y2, 5, 5) , col='black', cex=0.5)
dev.copy(png,"plot4_2_Y.png");
dev.off ();
rm(list = ls())

\end{lstlisting}
\pagebreak


\section{Question 1}

Box-Muller Method\\\\
For 100 values\\
Sample's Mean, X =  -0.1596442 , Y =  0.08031238 \\
Sample's Variance, X =  1.012568 , Y =  0.8934606 \\
Sample's Covariance =  -0.05055587 \\

For 500 values\\
Sample's Mean, X =  -0.09864711 , Y =  0.03518648 \\
Sample's Variance, X =  1.059539 , Y =  0.8675763 \\
Sample's Covariance =  -0.05837014 \\

For 10000 values\\
Sample's Mean, X =  -0.02360723 , Y =  0.01601492 \\
Sample's Variance, X =  0.9678944 , Y =  0.9818059 \\
Sample's Covariance =  -0.01037898 \\

\includegraphics{"plot1_1"}
\pagebreak


\includegraphics{"plot1_1_X"}
\pagebreak

\includegraphics{"plot1_1_Y"}
\pagebreak


\includegraphics{"plot1_2"}
\pagebreak

\includegraphics{"plot1_2_X"}
\pagebreak

\includegraphics{"plot1_2_Y"}
\pagebreak


\includegraphics{"plot1_3"}
\pagebreak

\includegraphics{"plot1_3_X"}
\pagebreak

\includegraphics{"plot1_3_Y"}

\pagebreak



Marsaglia-Bray Method\\\\

For 100 values\\
Sample's Mean, X =  0.02650457 , Y =  0.08917513 \\
Sample's Variance, X =  1.036414 , Y =  0.9381411 \\
Sample's Covariance =  0.03333995 \\

For 500 values\\
Sample's Mean, X =  0.01449956 , Y =  0.02634573 \\
Sample's Variance, X =  0.9660087 , Y =  0.8949109 \\
Sample's Covariance =  -0.1012995 \\

For 10000 values\\
Sample's Mean, X =  -0.002328363 , Y =  0.09431695 \\
Sample's Variance, X =  0.9546331 , Y =  0.9437196 \\
Sample's Covariance =  -0.03138388 \\

\includegraphics{"plot2_1"}
\pagebreak

\includegraphics{"plot2_1_X"}
\pagebreak

\includegraphics{"plot2_1_Y"}
\pagebreak


\includegraphics{"plot2_2"}
\pagebreak

\includegraphics{"plot2_2_X"}
\pagebreak

\includegraphics{"plot2_2_Y"}
\pagebreak


\includegraphics{"plot2_3"}
\pagebreak

\includegraphics{"plot2_3_X"}
\pagebreak

\includegraphics{"plot2_3_Y"}

\pagebreak


\section{Question 2}
$N(\mu= 0, \sigma=5)$ for 500 values\\\\
By Box-Muller\\

\includegraphics{"plot3_1"}
\pagebreak

\includegraphics{"plot3_1_X"}
\pagebreak

\includegraphics{"plot3_1_Y"}
\pagebreak


By Marsaglia-Bray
\includegraphics{"plot4_1"}
\pagebreak

\includegraphics{"plot4_1_X"}
\pagebreak

\includegraphics{"plot4_1_Y"}
\pagebreak


$N(\mu= 5, \sigma=5)$ for 500 values\\\\
By Box-Muller
\includegraphics{"plot3_2"}
\pagebreak

\includegraphics{"plot3_2_X"}
\pagebreak

\includegraphics{"plot3_2_Y"}
\pagebreak


By Marsaglia-Bray\\
\includegraphics{"plot4_2"}
\pagebreak

\includegraphics{"plot4_2_X"}
\pagebreak

\includegraphics{"plot4_2_Y"}
\pagebreak


\section{Question 3}
Box-Muller Method\\
Time taken for 10000 numbers generated =  0.01352119 \\

Marsaglia-Bray Method\\
Time taken for 10000 numbers generated =  0.009799719 \\

So, we se that Marsaglia-Bray Method is faster than Box-Muller Method.\\


\section{Question 4}
In Marsaglia-Bray Method\\
Rejection ratio, Theoretical =  0.2146018 , Generated =  0.2107963 \\
Simulated rejection ratio is very close to theoretical one.
\end{document}
